\begin{abstract}
  When Satoshi Nakamoto introduced Bitcoin, a central tenet was
  that the blockchain functions as a \emph{timestamping server}.
  In the Ethereum era, smart contracts widely assume on-chain timestamps
  are mostly accurate. In this paper, we prove this is indeed the case,
  namely that recorded timestamps do not wildly deviate from real-world
  time, a property we call \emph{timeliness}.
  We prove that all popular mechanisms for constructing permissionless
  blockchains (proof-of-work, longest chain proof-of-stake, and
  quorum-based proof-of-stake) are \emph{timely} under honest majority,
  but a synchronous network is a necessary condition.
  Next, we show that, given a global clock,
  all timely blockchains can be suitably modified,
  in a black-box fashion,
  such that all honest parties output exactly the same ledgers at
  the same round, achieving a property we call \emph{supersafety},
  which may be of independent interest.
  Conversely, we also show that supersafety implies timeliness,
  completing the circle.
\end{abstract}
