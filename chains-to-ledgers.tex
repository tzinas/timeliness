\subsection{From Chains to Ledgers}

We first define temporal blockchain protocols,
a subset of blockchain protocols which keep track of time.

\begin{definition}[(Temporal) Blockchain Protocol]
  A \emph{blockchain} protocol is distributed protocol
  in which, at the end of every round, each honest party outputs
  a chain. A chain is a finite sequence of blocks. Each block
  contains a finite sequence of transactions.

  A \emph{temporal blockchain} protocol is a blockchain protocol
  where every block contains a recorded round.
\end{definition}

\atnote{Refactor all C into $\Chain$ format.}
In blockchain protocols, honest parties hold at all times a finalized (or confirmed) chain
based on their confirmation rule\footnote{
  In longest-chain protocols, the longest observed chain without the last $k$ blocks
  is considered final.
  In Streamlet, the chain ending at the second of three consecutive notarized
  blocks with consecutive epochs is considered final.
}.
We use $\Chain[][P][r]$ to denote the finalized chain
of party $P$ at the end of round $r$.

\begin{definition}[Chain Safety]
  A blockchain protocol is \emph{safe} if for
  all honest parties $P_1, P_2$ and all rounds $r_1, r_2$,
  it holds that $\Chain[][P_1][r_1] \sim \Chain[][P_2][r_2]$.
  Furthermore, $\Chain[][P_1][r_1] \preccurlyeq \Chain[][P_1][r_2]$ (sticky).
\end{definition}

\noindent
\textbf{From Temporal Blockchains to Temporal Ledgers.}
Any temporal blockchain protocol can be transformed into a
temporal ledger protocol using the following construction:
When \rread is invoked on party $P$ at the end of round $r$, each transaction of
$\Chain[][P][r]$ is reported to $\Ledger[][P][r]$ with the same order, and with
the recorded round of the block in which it is included.
We call these protocols \emph{chain-based temporal ledger protocols}.
Ledger protocols based on safe blockchains are safe.

We now introduce two intermediate properties of temporal blockchains
that will help us prove timeliness of
chain-based temporal ledger protocols.

\begin{definition}[Consistent Recorded Rounds]
  A temporal blockchain protocol has \emph{consistent recorded rounds}
  when for all honest parties $P$ and rounds $r$,
  the recorded rounds in $\Chain[][P][r]$ are non-decreasing and not
  in the future.
\end{definition}

\begin{definition}[Freshness] \label{def:tip-freshness}
  A temporal blockchain protocol execution is \emph{fresh}($w$) when for
  any round $r$, the recorded round
  $r^*$ of the tip of any honest party's finalized chain
  satisfies $r - r^* \leq w$.
\end{definition}

The notion of freshness tells us that the recorded round
of any finalized chain tip cannot be too old.

\begin{theorem}[Freshness to Timeliness] \label{thm:freshness-to-timeliness}
  A temporal ledger protocol based on a safe and fresh($w$)
  temporal chain protocol with consistent recorded rounds is timely with timeliness $v = w$.
\end{theorem}
\begin{proof}
  Requirements (1) and (2) of timeliness are directly satisfied from
  the consistent recorded rounds.
  We now prove (3).

  Consider any honest party $P$, and any rounds $r_1 \leq r_2$.
  Suppose, towards a contradiction, that $\Ledger[][P][r_2][|\Ledger[][P][r_1]|{:}]$
  contains a transaction $\tx$ with recorded round $r \leq r_1 - v$.
  Due to safety, it holds that $\chain^P_{r_1} \prec \chain^P_{r_2}$.
  Transaction $\tx$ is in some block $B \in \chain^P_{r_2}[|\chain^P_{r_1}|{:}]$.
  Let $B^*$ be the tip of $\chain^P_{r_1}$ with recorded round $r^*$.
  Because of freshness, it holds that $r_1 - r^* \leq w$.
  Therefore:

  \begin{align*}
    \makebox[0.5\textwidth][c]{$r - r^* \leq w - v$}         && \text{[$r \leq r_1 - v$]} \\
    \makebox[0.5\textwidth][c]{$r - r^* \leq 0$}             && \text{[$v = w$]}
  \end{align*}

  This is a contradiction because block $B$ extends a chain that contains $B^*$,
  and hence $r > r^*$.
  \Qed
\end{proof}

In the next subsections, we express Streamlet, Bitcoin and Ouroboros
as chain-based temporal ledger protocols (with the above properties) to
prove them timely.
