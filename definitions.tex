\section{Definitions}

The following definitions are first introduced in this work.
\atnote{Write something in between the definitions.}

\begin{definition}[Temporal Ledger]
  A \emph{temporal ledger} is a finite sequence of pairs $(r, \tx)$ where $\tx$ is
  a transaction, and $r$ is a \emph{round} indicating the time at which
  the transaction in question is recorded on the ledger.
\end{definition}

\begin{definition}[Distributed Temporal Ledger Protocol]
  A \emph{distributed temporal ledger protocol} is a distributed ledger protocol
  that when \rread is invoked, honest parties output temporal ledgers instead of traditional ledgers.
\end{definition}

\begin{definition}[Timely]\label{def:timely}
  A distributed temporal ledger protocol is \emph{timely}$(v)$
  if for all honest parties $P$ and rounds $r_1$ it holds that:

  \begin{enumerate}
    \item The rounds recorded in $\Ledger[P][][r_1]$ are non-decreasing.\label{def:timely-increasing}
    \item The round recorded at $\Ledger[P][][r_1][-1]$ is at most $r_1$.\label{def:timely-past}
    \item For all $r_1 \leq r_2$, the rounds recorded in $\Ledger[P][][r_2][|\Ledger[P][][r_1]|{:}]$ are
          newer than $r_1 - v$.\label{def:timely-chunk}
  \end{enumerate}
\end{definition}

Note that timeliness is orthogonal to safety and liveness. Protocols can have any, none, or all three
of the properties.

\begin{definition}[Perfectly Timely]
  We call a protocol \emph{perfectly timely} if it is timely with parameter $v = 0$.
\end{definition}

\begin{definition}[Supersafety]
  A distributed ledger protocol is \emph{supersafe} if it is sticky and
  for any honest parties $P_1, P_2$ and any round $r$, it holds that
  $\Ledger[P_1][][r] = \Ledger[P_2][][r]$.
\end{definition}

Note that all supersafe protocols are safe.
