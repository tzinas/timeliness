\section{Introduction}
Everyone trusts the timestamps recorded on the
blockchain are accurate. But why?

In the original Bitcoin paper~\cite{bitcoin}, Nakamoto refers to the
blockchain as a distributed timestamping server.
Applications rely on the precision of these recorded timestamps.
However, timestamps in adversarially produced blocks may be fabricated.
And yet, empirical evidence suggests reported timestamps roughly correspond
to real world time.

In this paper, we put forth the notion of \emph{timeliness}, formalizing the folklore
understanding that timestamps recorded on the blockchain cannot deviate arbitrarily
from real world time. In particular we define a timeliness parameter $v$ which bounds
this potential deviation. Under honest majority, we prove this virtue materializes in all three flavors of
permissionless blockchains, namely proof-of-work,
longest chain proof-of-stake, and quorum-based proof-of-stake, and calculate the
timeliness parameter.

Blockchain systems executed in networks with delay, even when assuming a global clock and
a synchronous network, allow honest parties to reach consensus, but not at exactly
the same time. We observe that any timely blockchain can be modified to achieve
consensus at the same exact global time, a property we call \emph{supersafety},
albeit with the introduction of a small confirmation delay.

\noindent
\textbf{Our contributions.}

\begin{enumerate}
  \item We define the notion of \emph{timeliness} of a blockchain system.
  \item We prove that three exemplary cases of blockchain protocols (Bitcoin, Streamlet, and Ouroboros)
        are timely and calculate their timeliness parameter.
  \item We prove that timeliness cannot be achieved in partial synchrony.
  \item We present a black-box reduction from \emph{timeliness} to \emph{supersafety} and back,
        remaining mindful of late-joining clients.
\end{enumerate}

\noindent
\textbf{Related work.}
% - Variable difficulty bitcoin (lemma which states that difficulty estimation will be the roughly same for all honest parties?)
% - Ouroboros (construction which references the previous epoch to produce next epoch randomness by slot numbers)
% - Ouroboros Chronos
% - Ouroboros Klepsydra (this has a different title now, ask Srivatsan)
