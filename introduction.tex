\section{Introduction}
In the original Bitcoin paper~\cite{bitcoin}, Nakamoto refers to the
blockchain as a distributed timestamping server. However, when talking about
\emph{distributed} ledgers, we often neglect that, and only focus on the ordering
of transactions.
A ledger, namely a sequence of transactions, does not contain any information
about the time that each transaction was recorded. However, it might be useful
if each transaction was accompanied by a timestamp, representing the moment
that it was recorded on the ledger. We call this a \emph{temporal ledger};
namely a sequence of pairs $(\timestamp,\tx)$.

Blockchain applications like Bitcoin, Ethereum and Cardano
can all output temporal ledgers instead of regular ones. Only
minor client changes are required.

